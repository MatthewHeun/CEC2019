\title{\textbf{A survey of Christian perspectives \\
		on sustainability and related topics}}
\author{
		% Authors must be italicized.
        \emph{Jeremy Van Antwerp} and \emph{Matthew Kuperus Heun}%
\footnote{
Engineering Department, 
Calvin College,
Grand Rapids, MI, 49546, USA}
}

\documentclass[12pt]{article}

% CEC papers should be set in Times New Roman.
% https://tex.stackexchange.com/questions/153168/how-to-set-document-font-to-times-new-roman-by-command
% suggests the following.
\usepackage{mathptmx}             % For Times New Roman font (or a close approximation thereof).
\usepackage[margin=1in]{geometry} % For 1-inch margins all around.
\usepackage[document]{ragged2e}   % For left justification.
\usepackage{parskip}              % For double-spacing between paragraphs.
\usepackage{nopageno}             % To eliminate page numbers.
% For MLA bibliography. See https://www.ctan.org/pkg/biblatex-mla?lang=en for details.
\usepackage[american]{babel} 
\usepackage{csquotes} 
\usepackage[style=mla-new]{biblatex}
\addbibresource{JVAMKH.bib}
\usepackage{titlesec}             % To change formats of section titles, etc.
\titleformat*{\section}{\normalsize}
\titleformat*{\subsection}{\normalsize}
\titleformat*{\subsubsection}{\normalsize}
\usepackage{abstract}             % Set characteristics of the abstract
\setlength{\absleftindent}{0in}   % Do not indent left side
\setlength{\absrightindent}{0in}  % Do not indent right side
\usepackage{url}
\setlength{\parindent}{0in}       % Do not indent the 1st line of paragraphs.
\setlength{\parskip}{12pt}        % Instead, add space between paragraphs.
\renewcommand{\abstractnamefont}{\normalfont\normalsize} % Unbold and regular size
\renewcommand{\abstracttextfont}{\normalfont\normalsize} % Unbold and regular size
\date{}                           % To eliminate the date in the title

% Commands for editing.

\usepackage{xcolor}            % For colored text
\usepackage[normalem]{ulem}    % For \sout command (strikethrough)

% From https://tex.stackexchange.com/questions/130623/crossing-out-using-different-colour,
% Change the \sout color to red
\newcommand{\redsout}{\bgroup\markoverwith{\textcolor{red}{\rule[0.5ex]{2pt}{0.4pt}}}\ULon}

% Use these versions to display changes.
\newcommand{\del}[1]{\textcolor{gray}{\redsout{#1}}}
\newcommand{\ins}[1]{\textcolor{red}{#1}}
\newcommand{\rev}[2]{\del{#1}\ins{#2}}

% Use these versions for a clean copy.
% \newcommand{\del}[1]{}
% \newcommand{\ins}[1]{#1}
% \newcommand{\rev}[2]{#2}



\begin{document}
	
\maketitle

\begin{abstract}
\noindent
%**** This is the abstract that I have on file. ---MKH ****
This paper presents a survey of Christian perspectives, from ancient to modern, on a
variety of sustainability-related topics such as stewardship of the natural world,
economic growth and technological change, energy, and human population. The emphasis is on
the development of thought and the lineage of ideas, with application to modern viewpoints
on current issues such as climate change, sustainable development, and human migration. We
use Willis Jenkins’ topology (Ecologies of Grace) to organize three modern strands
(ecojustice, stewardship, and “ecological spiritualities”) arising from three Christian
traditions (Roman Catholicism, reformed Christianity, and Eastern Orthodoxy,
respectively).

The paper will cover a broad range of Christian thought and teaching in a digestible and
coherent format. It will serve as a supplement to a future engineering textbook on
sustainability challenges. Textbook chapters will provide a platform of background
knowledge to facilitate one-hour in-class discussions of several sustainability topics or
challenges. The conference presentation will highlight one area of Christian thought
(stewardship) and focus on piloting classroom discussion questions related to the theology
of sustainability.


% !!!! OBJECTIVE IS TO ACCURATELY SUMMARIZE THE FULL RANGE OF CHRISTIAN THOUGHT ON SUSTAINABILITY ISSUES.
% !!!  FOCUS IS ON 
% !!!! STRESS COMMONALITIES, NOTE DIFFERENCES

\end{abstract}


%%%%%%%%%%%%%%%%%%%%%%%%%%%%%%%%%%%%%%%%%%%%%%%%%%%%%%%%%%%%%%
\section{Introduction}
\label{sec:introduction}
%%%%%%%%%%%%%%%%%%%%%%%%%%%%%%%%%%%%%%%%%%%%%%%%%%%%%%%%%%%%%%

Example citation~\autocite[19]{Jenkins:2008}.

Sustainability is a popular topic and is of increasing concern in the modern world. This
paper will begin to answer the questions ``how have Christians in the past thought about
sustainability topics?'' ```How do Christians today think about sustainability topics?''
and ``Where do these views come from?'' We will end with some questions for thought and
discussion.

Sustainability is often organized into three categories: environmental, economic, and
social. Environmental sustainability involves the idea of preserving the natural world.
Christians have historically understood this as stewardship of the natural world, going
back to the charge God gives humankind in Genesis 1:26. Economic sustainability refers to
the preservation and increase in value of human activities. In other words, not everything
we do can lose money. Social sustainability refers to relationships among humans -- that
there should be justice, peace, order, and flourishing in human society. For Christians,
the root of this concern is the command to ``love your neighbor as yourself.'' Overarching
these three areas are concerns for human shalom and wellbeing.

How Christians have understood these imperatives has varied considerably over millennia. 
We talked about survey versus introduction and how to handle references.

%Paragraphs must be separated by a blank line.
%This paragraph has a citation~\autocite[42]{bogus}.


%1%%%%%%%%%%%%%%%%%%%%%%%%%%%%%%%%%%%%%%%%%%%%%%%%%%%%%%%%%%%%
\section{Environmental}
\label{sec:environmental}
%%%%%%%%%%%%%%%%%%%%%%%%%%%%%%%%%%%%%%%%%%%%%%%%%%%%%%%%%%%%%%

Natural world is inherently good.
Natural world should be conquered and ruled over (in part, because it is `red in tooth and claw.')

%Include table with links to denominational statements on climate change, whaling, conservation, pollution, wilderness?

*** Here is a URL that shows how to convert a .csv file into a LaTeX table.

\url{https://www.latex-tutorial.com/tutorials/pgfplotstable/}

Here is a link to the Google sheet.

{\tiny \url{https://docs.google.com/spreadsheets/d/1OD7LFbCU0Rdv3-vAM0MbIBd4hkEwxwY2FtP0XfvAU_M/edit#gid=0}}



%++++++++++++++++++++++++++++++
\subsection{Environmental history}
\label{sec:environmental_history}
%++++++++++++++++++++++++++++++


%2%%%%%%%%%%%%%%%%%%%%%%%%%%%%%%%%%%%%%%%%%%%%%%%%%%%%%%%%%%%%
\section{Economic}
\label{sec:economic}
%%%%%%%%%%%%%%%%%%%%%%%%%%%%%%%%%%%%%%%%%%%%%%%%%%%%%%%%%%%%%%
Economics and politics (the ``social'' ) are inextricably linked.

The Bible has a lot to say about money. From the first chapter of Genesis, the call to
stewardship of all creation has been understood by Christians to include money. The power
of earthly resources to accomplish ``heavenly things'' is made explicit in the parable of
the shrewd (or dishonest) manager in Luke 16. To this end, the great majority of the
Bible's teaching on money relates to generosity to the poor. Numerous Old and New
Testament passages instruct us to give generously 
\ins{to what? To alleviate poverty? To the church?  And who is ``us''?}.

One branch of Christian thought sees riches themselves as a root of evil. 
This \ins{antecedent?} goes beyond
merely \emph{love} money being the root of evil (I Tim 6:10). Proponents of this view
\ins{(which view? Riches themselves being the root of all evil?
or the love of money being the root of all evil?)}
would point to Jesus telling the rich young man to sell all his possessions and give to
the poor and Jesus' further comment that it is easier for a camel to go through the eye of
a needle than for a rich man to enter the kingdom of God (Mt 19:16-30, Mk 10:17-31, Luke
18:18-30).

At the other end of the spectrum of Christian thought, worldly wealth is seen as God's
blessing, even an indication of his favor in a more extreme version of this view.

\ins{This paragraph seems like a contrast that you didn't set up earlier. 
Perhaps earlier you should say ``Typically, the Bible stays away from 
what we now would call an ``economic theory''. 
Rather, scripture provides isolated, micro-level indications of how individuals
should relate to each other on economic matters.''
The setup could come right after ``The Bible has a lot to say about money.''}
With one exception, the Bible does not present a full economic \emph{system} 
in the sense that we would use the term today.
That exception is the Old Testament system of canceling debt and returning property in the
year of Jubilee. However, it is not clear to what extent this model system was ever used in practice.
In modern economic terms, canceling debts and returning property would serve to minimize 
\emph{income inequality} and ensure that there was uniform access to the \emph{means of production}.

In terms of modern economic views, Christians have advocated positions ranging from
communism (albeit not in a political sense), for example monastic orders or the Hutterite
and Bruderhof communities, to capitalism.

\begin{itemize}
\item{What are the economic sustainability issues we want to address? How do we want to address them?}
\item{Global poverty and other ills associated with it like lack of access to clean water and food insecurity.}
\item{The concept of continual economic growth based on extractive consumption.}
\item{Carbon tax}
\end{itemize}
%
%Wealth is a sign of God's blessing. 
%Personal property. Western of ``Christian'' concepts.
%The Enlightenment and the concept of personal freedom.
%Development of corporations.
%Theology of saving. Commerce, banking, financial trading.
%Roman Catholic response(s) to communism?
%Who owns land? Labor? Capital?
%
%% Quote from Wikipedia on Distributionism 
%% https://en.wikipedia.org/wiki/Distributism
%Distributism is an economic ideology asserting that the world's productive assets should
%be widely owned rather than concentrated.[1] It was developed in Europe in the late 19th
%and early 20th centuries based upon the principles of Catholic social teaching, especially
%the teachings of Pope Leo XIII in his encyclical Rerum novarum (1891) and Pope Pius XI in
%Quadragesimo anno (1931).[2][3][4] It views both capitalism and socialism as equally
%flawed and exploitative, and it favors economic mechanisms such as small-scale
%cooperatives and family businesses, and large-scale anti-trust regulations.
%
%Some Christian Democratic political parties have advocated distributism in their economic
%policies.
% end quote

% In this issue, Nathan Schneider writes about Pope Francis’s economics. Here, he
% recommends five books of Catholic thought that display strikingly similar concerns to
% those of secular activists today. Each one emphasizes the wisdom of ordinary people. “In
% church each week,” Schneider says, “I learn Catholic economics from the diversity of
% classes and colors who meet under the image of an executed radical.”
%  https://www.thenation.com/article/classics-of-catholic-economics/

%++++++++++++++++++++++++++++++
\subsection{History of Christian economic thought}
\label{sec:economic_history}
%++++++++++++++++++++++++++++++



%3%%%%%%%%%%%%%%%%%%%%%%%%%%%%%%%%%%%%%%%%%%%%%%%%%%%%%%%%%%%%
\section{Social}
\label{sec:social}
%%%%%%%%%%%%%%%%%%%%%%%%%%%%%%%%%%%%%%%%%%%%%%%%%%%%%%%%%%%%%%
Plot economic and social positions on orthogonal axes?
\begin{itemize}
\item{https://en.wikipedia.org/wiki/Political\_compass}
\item{https://en.wikipedia.org/wiki/Nolan\_Chart}
\item{https://en.wikipedia.org/wiki/Pournelle\_chart}
\end{itemize}
All three of these have libertarian connections that I don't care for. Maybe abandon the idea?

Suffering and justice. Care for the poor. 
Kings, democracy, dictators.
Medicine, water, health care. What are ``rights?''
What are the means of redress?
Just war? just use of ``power?''
Sphere sovereignty and the role of government? Church? family ... school...


%++++++++++++++++++++++++++++++
\subsection{History of Christian social thought}
\label{sec:social_history}
%++++++++++++++++++++++++++++++


\section{Other?}
\section{Conclusions?}



\printbibliography


\end{document}