\title{\textbf{A survey of Christian perspectives \\
		on sustainability and related topics}}
\author{
		% Authors must be italicized.
        \emph{Jeremy Van Antwerp}, \emph{Matthew Kuperus Heun}%
\footnote{
Engineering Department, 
Calvin College,
Grand Rapids, MI, 49546, USA}
}

\documentclass[12pt]{article}

% CEC papers should be set in Times New Roman.
% https://tex.stackexchange.com/questions/153168/how-to-set-document-font-to-times-new-roman-by-command
% suggests the following.
\usepackage{mathptmx} % For Times New Roman font (or a close approximation thereof).
\usepackage[margin=1in]{geometry} % For 1-inch margins all around.
\usepackage[document]{ragged2e} % For left justification.
\usepackage{parskip} % For double-spacing between paragraphs.
\usepackage{nopageno} % To eliminate page numbers.
% For MLA bibliography. See https://www.ctan.org/pkg/biblatex-mla?lang=en for details.
\usepackage[american]{babel} 
\usepackage{csquotes} 
\usepackage[style=mla-new]{biblatex}
\addbibresource{JVAMKH.bib}
\usepackage{titlesec} % To change formats of section titles, etc.
\titleformat*{\section}{\normalsize}
\titleformat*{\subsection}{\normalsize}
\titleformat*{\subsubsection}{\normalsize}
\usepackage{abstract} % Set the characteristics of the abstract
\setlength{\absleftindent}{0in}
\setlength{\absrightindent}{0in}
\setlength{\parindent}{0in} % Do not indent the 1st line of a paragraph.
\setlength{\parskip}{3mm}   % Add space between paragraphs.
\renewcommand{\abstractnamefont}{\normalfont\normalsize}
\renewcommand{\abstracttextfont}{\normalfont\normalsize}


\begin{document}
	
\maketitle

\begin{abstract}
\noindent
**** This is the abstract that I have on file. ---MKH ****
This paper presents a survey of Christian perspectives, 
from ancient to modern, 
on a variety of sustainability-related topics such as stewardship of the natural world, 
economic growth and technological change, energy, and human population. 
The emphasis is on the development of thought and the lineage of ideas, 
with application to modern viewpoints on current issues 
such as climate change, sustainable development, and human migration. 
We use Willis Jenkins’ topology (Ecologies of Grace) 
to organize three modern strands 
(ecojustice, stewardship, and “ecological spiritualities”) 
arising from three Christian traditions 
(Roman Catholicism, reformed Christianity, and Eastern Orthodoxy, respectively).

The paper will cover a broad range of Christian thought and teaching 
in a digestible and coherent format. 
It will serve as a supplement to a future engineering textbook on sustainability challenges. 
Textbook chapters will provide a platform 
of background knowledge to facilitate one-hour in-class discussions
of several sustainability topics or challenges. 
The conference presentation will highlight 
one area of Christian thought (stewardship) 
and focus on piloting classroom discussion questions related to the theology of sustainability.
\end{abstract}


%%%%%%%%%%%%%%%%%%%%%%%%%%%%%%%%%%%%%%%%%%%%%%%%%%%%%%%%%%%%%%
\section{Introduction}
\label{sec:introduction}
%%%%%%%%%%%%%%%%%%%%%%%%%%%%%%%%%%%%%%%%%%%%%%%%%%%%%%%%%%%%%%

Introduction goes here. 
More information. 
The next sentence. 
Have I wrapped to the next line yet?

Paragraphs must be separated by a blank line.
This paragraph has a citation~\autocite[42]{bogus}.


%%%%%%%%%%%%%%%%%%%%%%%%%%%%%%%%%%%%%%%%%%%%%%%%%%%%%%%%%%%%%%
\section{Next section}
\label{sec:next_sect}
%%%%%%%%%%%%%%%%%%%%%%%%%%%%%%%%%%%%%%%%%%%%%%%%%%%%%%%%%%%%%%

Contains more text.


%++++++++++++++++++++++++++++++
\subsection{Subsection}
\label{sec:subsection_label}
%++++++++++++++++++++++++++++++


\printbibliography


\end{document}