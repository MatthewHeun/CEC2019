\title{\textbf{A survey of Christian perspectives \\
		on sustainability and related topics}}
\author{
		% Authors must be italicized.
        \emph{Jeremy Van Antwerp} and \emph{Matthew Kuperus Heun}%
\footnote{
Engineering Department, 
Calvin College,
Grand Rapids, MI, 49546, USA}
}

\documentclass[12pt]{article}

% CEC papers should be set in Times New Roman.
% https://tex.stackexchange.com/questions/153168/how-to-set-document-font-to-times-new-roman-by-command
% suggests the following.
\usepackage{mathptmx}             % For Times New Roman font (or a close approximation thereof).
\usepackage[margin=1in]{geometry} % For 1-inch margins all around.
\usepackage[document]{ragged2e}   % For left justification.
\usepackage{parskip}              % For double-spacing between paragraphs.
\usepackage{nopageno}             % To eliminate page numbers.
% For MLA bibliography. See https://www.ctan.org/pkg/biblatex-mla?lang=en for details.
\usepackage[american]{babel} 
\usepackage{csquotes} 
\usepackage[style=mla-new]{biblatex}
\addbibresource{JVAMKH.bib}
\usepackage{titlesec}             % To change formats of section titles, etc.
\titleformat*{\section}{\normalsize}
\titleformat*{\subsection}{\normalsize}
\titleformat*{\subsubsection}{\normalsize}
\usepackage{abstract}             % Set characteristics of the abstract
\setlength{\absleftindent}{0in}   % Do not indent left side
\setlength{\absrightindent}{0in}  % Do not indent right side
\usepackage{url}
\setlength{\parindent}{0in}       % Do not indent the 1st line of paragraphs.
\setlength{\parskip}{12pt}        % Instead, add space between paragraphs.
\renewcommand{\abstractnamefont}{\normalfont\normalsize} % Unbold and regular size
\renewcommand{\abstracttextfont}{\normalfont\normalsize} % Unbold and regular size
\date{}                           % To eliminate the date in the title

% Commands for editing.

\usepackage{xcolor}            % For colored text
\usepackage[normalem]{ulem}    % For \sout command (strikethrough)

% From https://tex.stackexchange.com/questions/130623/crossing-out-using-different-colour,
% Change the \sout color to red
\newcommand{\redsout}{\bgroup\markoverwith{\textcolor{red}{\rule[0.5ex]{2pt}{0.4pt}}}\ULon}

% Use these versions to display changes.
\newcommand{\del}[1]{\textcolor{gray}{\redsout{#1}}}
\newcommand{\ins}[1]{\textcolor{red}{#1}}
\newcommand{\rev}[2]{\del{#1}\ins{#2}}

% Use these versions for a clean copy.
% \newcommand{\del}[1]{}
% \newcommand{\ins}[1]{#1}
% \newcommand{\rev}[2]{#2}



\begin{document}
	
\maketitle

\begin{abstract}
\noindent
%**** This is the abstract that I have on file. ---MKH ****
This paper presents a survey of Christian perspectives, from ancient to modern, on a
variety of sustainability-related topics such as stewardship of the natural world,
economic growth and technological change, energy, and human population. The emphasis is on
the development of thought and the lineage of ideas, with application to modern viewpoints
on current issues such as climate change, sustainable development, and human migration. We
use Willis Jenkins’ topology (Ecologies of Grace) to organize three modern strands
(ecojustice, stewardship, and “ecological spiritualities”) arising from three Christian
traditions (Roman Catholicism, reformed Christianity, and Eastern Orthodoxy,
respectively).

The paper will cover a broad range of Christian thought and teaching in a digestible and
coherent format. It will serve as a supplement to a future engineering textbook on
sustainability challenges. Textbook chapters will provide a platform of background
knowledge to facilitate one-hour in-class discussions of several sustainability topics or
challenges. The conference presentation will highlight one area of Christian thought
(stewardship) and focus on piloting classroom discussion questions related to the theology
of sustainability.


% !!!! OBJECTIVE IS TO ACCURATELY SUMMARIZE THE FULL RANGE OF CHRISTIAN THOUGHT ON SUSTAINABILITY ISSUES.
% !!!  FOCUS IS ON 
% !!!! STRESS COMMONALITIES, NOTE DIFFERENCES

\end{abstract}


%%%%%%%%%%%%%%%%%%%%%%%%%%%%%%%%%%%%%%%%%%%%%%%%%%%%%%%%%%%%%%
\section{Introduction}
\label{sec:introduction}
%%%%%%%%%%%%%%%%%%%%%%%%%%%%%%%%%%%%%%%%%%%%%%%%%%%%%%%%%%%%%%

%Example citation~\autocite[19]{Jenkins:2008}.

%Sustainability is a popular topic and is of increasing concern in the modern world. This
%paper will begin to answer the questions ``how have Christians in the past thought about
%sustainability topics?'' ```How do Christians today think about sustainability topics?''
%and ``Where do these views come from?'' We will end with some questions for thought and
%discussion.

The topic of sustainability is often organized into three categories: environmental, economic, and
social. Environmental sustainability involves the idea of preserving the natural world.
Christians have historically understood this as stewardship of the natural world, going
back to the charge God gives humankind in Genesis 1:26. Economic sustainability refers to
the preservation and increase in value of human activities. In other words, not everything
we do can lose money. Social sustainability refers to relationships among humans -- that
there should be justice, peace, order, and flourishing in human society. For Christians,
the root of this concern is the command to ``love your neighbor as yourself.'' Overarching
these three areas are concerns for human shalom and wellbeing.

How Christians have understood these imperatives has varied considerably over millennia. 
%We talked about survey versus introduction and how to handle references.
This paper will explore the questions ``how do Christians think about
sustainability topics?'' and ``Where do their views come from?'' 
We use, as a case study, the idea of banning personal vehicles.

%Paragraphs must be separated by a blank line.
%This paragraph has a citation~\autocite[42]{bogus}.


%1%%%%%%%%%%%%%%%%%%%%%%%%%%%%%%%%%%%%%%%%%%%%%%%%%%%%%%%%%%%%
\section{Environmental}
\label{sec:environmental}
%%%%%%%%%%%%%%%%%%%%%%%%%%%%%%%%%%%%%%%%%%%%%%%%%%%%%%%%%%%%%%

Willis Jenkins' book \emph{Ecologies of Grace}~\autocite{Jenkins:2008}
provides a topology of Christian thought regarding the 
nonhuman creation and the
environmental aspect of sustainability.
He identifies three schools of thought:
stewardship, 
eco-justice, and 
ecological spirituality,
which loosely correspond to 
Reformed (or evangelical protestant), 
Roman Catholic, and 
Eastern Orthodox 
traditions.
To Jenkins' three, we add a fourth:
consumptive economic prosperity and
its conservative evangelical tradition. 
These four schools of thought 
span a wide range of Christian responses to the nonhuman creation
and 
consequently outline a range of possibilities 
for Christian responses to environmental issues
and environmental sustainability concerns more broadly.
One way to begin unpacking the four schools of thought 
is to identify a keyword for each:
\emph{redemption} for stewardship, 
\emph{sanctification} for eco-justice,
\emph{deification} for ecological spirituality, and
\emph{resilience} for consumptive economic prosperity.

%..............................
\paragraph{Redemption} 
\label{sec:redemption}
%..............................

The stewardship school of thought in the Reformed tradition
emphasizes that all of human existence
is a response to God's redemptive acts
and God's providence to humans.
Knowing God leads to vocational responsibility 
to care for nonhuman creation,
which is the means by which God provides for humankind \autocite[19]{Jenkins:2008}. 
Thus, all human work to care for the creation 
is seen as service to the Creator
out of gratitude for redemption \autocite[77]{Jenkins:2008}.

%..............................
\paragraph{Sanctification} 
\label{sec:sanctification}
%..............................

The eco-justice school of thought in the Roman Catholic tradition
emphasizes that God's grace reveals the creation's 
inherent integrity \autocite[19]{Jenkins:2008}, 
giving it natural value and inherent moral standing~\autocite{Joldersma:2019}. 
Thus, Christians must respect creation's inherent value and 
respond to its moral standing in all activities.
If the nonhuman creation can't speak for itself, 
we must speak for it and defende it where necessary.

%..............................
\paragraph{Deification} 
\label{sec:deification}
%..............................

The deification school of thought in the Eastern Orthodox tradition
highlights the union between all of creation and God.
Thus, there is a radical, integral relationship between humans and 
the nonhuman creation~\autocite[93]{Jenkins:2008}.

%..............................
\paragraph{Resilience} 
\label{sec:resilience}
%..............................

The resilience school of thought in the conservative evangelical tradition
holds that the nonhuman creation is resilient, robust, and self-correcting.
Furthermore, its well-being is assured by God.
In addition, human well-being is paramount. 
Thus, humans are to be consumers of the nonhuman creation 
to provide economic prosperity and
lift people out of poverty~\autocite{Cornwall:2006aa}.


% Natural world is inherently good.
% Natural world should be conquered and ruled over (in part, because it is `red in tooth and claw.')

%Include table with links to denominational statements on climate change, whaling, conservation, pollution, wilderness?

% *** Here is a URL that shows how to convert a .csv file into a LaTeX table.
%
% \url{https://www.latex-tutorial.com/tutorials/pgfplotstable/}
%
% Here is a link to the Google sheet.
%
% {\tiny \url{https://docs.google.com/spreadsheets/d/1OD7LFbCU0Rdv3-vAM0MbIBd4hkEwxwY2FtP0XfvAU_M/edit#gid=0}}



%++++++++++++++++++++++++++++++
\subsection{Environmental history}
\label{sec:environmental_history}
%++++++++++++++++++++++++++++++


%2%%%%%%%%%%%%%%%%%%%%%%%%%%%%%%%%%%%%%%%%%%%%%%%%%%%%%%%%%%%%
\section{Economic}
\label{sec:economic}
%%%%%%%%%%%%%%%%%%%%%%%%%%%%%%%%%%%%%%%%%%%%%%%%%%%%%%%%%%%%%%
Economics and politics (the ``social'' ) are inextricably linked.

The Bible has a lot to say about money. From the first chapter of Genesis, the call to
stewardship of all creation has been understood by Christians to include money. The power
of earthly resources to accomplish ``heavenly things'' is made explicit in the parable of
the shrewd (or dishonest) manager in Luke 16. To this end, the great majority of the
Bible's teaching on money relates to generosity to the poor. Numerous Old and New
Testament passages instruct us to give generously 
\ins{to what? To alleviate poverty? To the church?  And who is ``us''?}.

One branch of Christian thought sees riches themselves as a root of evil. 
This \ins{antecedent?} goes beyond
merely \emph{love} money being the root of evil (I Tim 6:10). Proponents of this view
\ins{(which view? Riches themselves being the root of all evil?
or the love of money being the root of all evil?)}
would point to Jesus telling the rich young man to sell all his possessions and give to
the poor and Jesus' further comment that it is easier for a camel to go through the eye of
a needle than for a rich man to enter the kingdom of God (Mt 19:16-30, Mk 10:17-31, Luke
18:18-30).

At the other end of the spectrum of Christian thought, worldly wealth is seen as God's
blessing, even an indication of his favor in a more extreme version of this view.

\ins{This paragraph seems like a contrast that you didn't set up earlier. 
Perhaps earlier you should say ``Typically, the Bible stays away from 
what we now would call an ``economic theory''. 
Rather, scripture provides isolated, micro-level indications of how individuals
should relate to each other on economic matters.''
The setup could come right after ``The Bible has a lot to say about money.''}
With one exception, the Bible does not present a full economic \emph{system} 
in the sense that we would use the term today.
That exception is the Old Testament system of canceling debt and returning property in the
year of Jubilee. However, it is not clear \ins{(to whom?)} 
to what extent this model system was ever used in practice.
In modern economic terms, canceling debts and returning property would serve to minimize 
\emph{income inequality} and ensure that there was uniform access to the \emph{means of production}.

\ins{The above is all good. 
Next step is to connect to environmental sustainability.
Maybe identify some environmental sustainability questions/issues
and show how the different Christian positions on the economy
influence the response to the environmental issues?}

In terms of modern economic views, Christians have advocated positions ranging from
communism (albeit not in a political sense), for example monastic orders or the Hutterite
and Bruderhof communities, to capitalism.

\begin{itemize}
\item{What are the economic sustainability issues we want to address? How do we want to address them?}
\item{Global poverty and other ills associated with it like lack of access to clean water and food insecurity.}
\item{The concept of continual economic growth based on extractive consumption.}
\item{Carbon tax}
\end{itemize}
%
%Wealth is a sign of God's blessing. 
%Personal property. Western of ``Christian'' concepts.
%The Enlightenment and the concept of personal freedom.
%Development of corporations.
%Theology of saving. Commerce, banking, financial trading.
%Roman Catholic response(s) to communism?
%Who owns land? Labor? Capital?
%
%% Quote from Wikipedia on Distributionism 
%% https://en.wikipedia.org/wiki/Distributism
%Distributism is an economic ideology asserting that the world's productive assets should
%be widely owned rather than concentrated.[1] It was developed in Europe in the late 19th
%and early 20th centuries based upon the principles of Catholic social teaching, especially
%the teachings of Pope Leo XIII in his encyclical Rerum novarum (1891) and Pope Pius XI in
%Quadragesimo anno (1931).[2][3][4] It views both capitalism and socialism as equally
%flawed and exploitative, and it favors economic mechanisms such as small-scale
%cooperatives and family businesses, and large-scale anti-trust regulations.
%
%Some Christian Democratic political parties have advocated distributism in their economic
%policies.
% end quote

% In this issue, Nathan Schneider writes about Pope Francis’s economics. Here, he
% recommends five books of Catholic thought that display strikingly similar concerns to
% those of secular activists today. Each one emphasizes the wisdom of ordinary people. “In
% church each week,” Schneider says, “I learn Catholic economics from the diversity of
% classes and colors who meet under the image of an executed radical.”
%  https://www.thenation.com/article/classics-of-catholic-economics/

% from ENGR 184 lecture slides: 
``socially desirable and economically viable.''
Economic: profit, cost savings, economic growth, R\&D
Social: standard of living, education, community, equal opportunity. These combine to be business ethics, fair trade, women's rights.
Economics refers to the whole community impact, not just a company's bottom line.
social is fair business practices to labor and the community (whatever that means).
Economics is the social science that seeks to describe the factors which determine the production, distribution, and consumption of goods and services.
Food production, distribution, quality -- relates to land use and water consumption (3 models of agriculture: what's ``most efficient?'')


%++++++++++++++++++++++++++++++
\subsection{History of Christian economic thought}
\label{sec:economic_history}
%++++++++++++++++++++++++++++++



%3%%%%%%%%%%%%%%%%%%%%%%%%%%%%%%%%%%%%%%%%%%%%%%%%%%%%%%%%%%%%
\section{Social}
\label{sec:social}
%%%%%%%%%%%%%%%%%%%%%%%%%%%%%%%%%%%%%%%%%%%%%%%%%%%%%%%%%%%%%%

Plot economic and social positions on orthogonal axes?
\begin{itemize}
\item{https://en.wikipedia.org/wiki/Political\_compass}
\item{https://en.wikipedia.org/wiki/Nolan\_Chart}
\item{https://en.wikipedia.org/wiki/Pournelle\_chart}
\end{itemize}
All three of these have libertarian connections that I don't care for. Maybe abandon the idea?

Suffering and justice. Care for the poor. 
Kings, democracy, dictators.
Medicine, water, health care. What are ``rights?''
What are the means of redress?
Just war? just use of ``power?''
Sphere sovereignty and the role of government? Church? family ... school...


%++++++++++++++++++++++++++++++
\subsection{History of Christian social thought}
\label{sec:social_history}
%++++++++++++++++++++++++++++++


%%%%%%%%%%%%%%%%%%%%%%%%%%%%%%%%%%%%%%%%%%%%%%%%%%%%%%%%%%%%%%
\section{Application: personal transportation}
\label{sec:personal_transportation}
%%%%%%%%%%%%%%%%%%%%%%%%%%%%%%%%%%%%%%%%%%%%%%%%%%%%%%%%%%%%%%

% needs to somehow address the framework by which we make choices. I want to get at the utter failure of current and past modes of thinking
% in allowing us to address sustainability-related questions.
Far from being esoteric or ``merely'' philosophical, the impact of worldview on sustainability choices is both crucially important and entirely practical. 
% not to say that engineers can only study things that are ``practical.''
Thus, it is \emph{essencial} that engineers consider worldview when examining choices and tradeoffs related to stability.
This section considers, as a practical example, the effects of different Christian worldviews on a sustainability-driven policy.

In 2017, the United States emitted 5.14 billion metric tons equivalent of greenhouse gasses, of which, 
about 1.8 billion tons (35\%) were from petroleum used for transportation \cite{EIA2017}.
Consider a proposed ban on personal vehicles that use fossil fuels.
Personal vehicles that do not use fossil fuels, such as all-electric vehicles, would not be banned. 
% what's the source of electricity?
Likewise, commercial vehicles are not addressed by this proposal. % is that fair? what about rental cars?
Below we evaluate banning personal vehicles that consume fossil fuels from the perspectives of the Christian worldviews outlined in section \ref{sec:environmental}.


%++++++++++++++++++++++++++++++
\subsection{Evaluating tradeoffs}
\label{sec:evaluating_tradeoffs}
%++++++++++++++++++++++++++++++
Every decision involves tradeoffs. This section outlines some of the tradeoffs entailed in banning fossil-fueled personal transport vehicles 
and weighs them from the different worldviews outlined in section \ref{sec:environmental}.
That is, how do each of the Christian traditions discussed in this paper 
evaluate tradeoffs among environmental, social, and economic factors?

%..............................
\paragraph{Reformed (stewardship)} 
%..............................

Stewarding involves tradeoffs.  % this is, in an engineering sense, then a maximization? 
Each tradeoff should be weighed individually. % convert all choices to a common numeraire?
All voices should be heard (danger in not hearing voices). % what are the different voices?

%..............................
\paragraph{Roman Catholic (eco-justice)} 
%..............................

The moral standing of the nonhuman creation means that 
the creation itself must be given a voice. 
Evaluating tradeoffs means that someone must be empowered 
to speak for the nonhuman creation and 
give voice to injustice and unfairness. 
Humans will be persuaded by the “court” of judgment, 
to give the nonhuman creation its due.

%..............................
\paragraph{Eastern Orthodox (ecological spirituality)} 
%..............................

Because humans are radically connected to the nonhuman creation, 
any good done to the environment is a good done to us, 
and praiseworthy. 
Conversely, any environmental harm is a harm to ourselves and sinful. 
We should always be doing right by ourselves and the environment 
to please the God of us all.

%..............................
\paragraph{Conservative Evangelical (consumptive economic prosperity)} 
%..............................

Human economic prosperity will provide the resources 
to correct any environmental degradation, which, 
by the way, will never be so bad as to be unfixable, 
because the Earth is a resilient system. 
So, increasing monetary wealth is paramount. 
Any environmental damage caused by wealth creation will be fixable by the invisible hand. 


%++++++++++++++++++++++++++++++
\subsection{A conundrum}
\label{sec:conundrum}
%++++++++++++++++++++++++++++++

The proposed solution implicitly accepts a conundrum: 
you can't have personal transport vehicles and a flourishing environment. 

%..............................
\paragraph{What assumptions are behind that conundrum?} 
%..............................

The proposed solution assumes that all personal transport options degrade the environment. 

%..............................
\paragraph{Can you think of alternative policies that break free from the conundrum and solve the dilemma?} 
%..............................

Alternative policies include vehicles that can be fabricated and operated entirely by renewable energy that emits no CO2. 

%..............................
\paragraph{What are the social, economic, and environmental costs of those alternative policies?} 
%.............................. 

To emplace a fully-renewable transportation infrastructure would involve massive costs. 
Economically, billions of dollars per year would be required. 
One estimate is between 7 and 14\% of GDP annually 
over 20 years would be required (Reference: MKH’s ECON233 lectures). 
Emplacing a fully-renewable transportation infrastructure 
would displace or require thorough transformation of many existing industries, 
including filling stations, 
the automobile industry, 
electricity production and distribution industries, 
long-haul trucking and goods distribution, etc. 
Disruptions in those industries 
would cause massive social disruptions 
as jobs are lost and retraining would be required.

Also, there's the scale of mass transit, 
shared transit (zip cars, Lyft/Uber). 
There’s a land use question: 
the ``freedom'' of rural life vs.\ density of urban life. 
Maybe different rules for different places?
Techno-optimism (technocism?) 
posits there is a way to solve all problems with sufficiently advanced technology. 
What would be the (socialism?) way?---there's a way to solve every problem 
with a sufficiently advanced society.




%%%%%%%%%%%%%%%%%%%%%%%%%%%%%%%%%%%%%%%%%%%%%%%%%%%%%%%%%%%%%%
\section{Conclusions}
\label{sec:conclusions}
%%%%%%%%%%%%%%%%%%%%%%%%%%%%%%%%%%%%%%%%%%%%%%%%%%%%%%%%%%%%%%


\appendix

%%%%%%%%%%%%%%%%%%%%%%%%%%%%%%%%%%%%%%%%%%%%%%%%%%%%%%%%%%%%%%
\section{Discussion questions}
\label{sec:discussion_questions}
%%%%%%%%%%%%%%%%%%%%%%%%%%%%%%%%%%%%%%%%%%%%%%%%%%%%%%%%%%%%%%

%
\begin{enumerate}

  \item In Table~xx, columns give Christian traditions
        and rows provide aspects of their responses
		to environmental sustainability issues. 
		Apply your understanding of each tradition to 
		predict how each would respond to the following policy proposals
		for solving climate change issues:
		%
		\begin{itemize}

		  \item Tax fossil fuel \emph{production} and use the proceeds to 
		        convert all energy infrastructure to renewables
		  
		  \item Tax fossil fuel \emph{consumption} and use the proceeds to 
		        convert all energy infrastructure to renewables

		  \item Stimulate economic expansion to create excess wealth to be applied to 

		\end{itemize}
		%
  \item ``Environmental'' is only one dimension of sustainability. 
        Tradeoffs often exist among ``environmental,'' ``social,'' and ``financial'' 
		aspects of sustainability. 
		How would each Christian tradition in Table~xx respond to the following tradeoffs?
		%
		\begin{itemize}

		  \item Long-term employment losses due to climate-change-caused flooding vs.\ 
		        short-term employment retention for coal miners

		  \item

		\end{itemize}
		%
  \item Which Christian tradition resonates with you personally? Why?
  
  \item Which aspects of other Christian traditions do you embrace? Why?

  \item What is the (a) Christian approach to thinking about tradeoffs 
        between long-term versus short-term effects?
		
  \item Likewise, what's a Christian approach to thinking about tradeoffs in different areas
        (for instance, human well being versus the environment)?

  \item What is the purpose of the Creation? The natural world?

  \item What is aspects of economic sustainability are most important to a Christian?


\item The transportation system provides economic value and connects people. But it causes CO2 to be emitted. We must reduce the CO2 intensity of travel to move toward a sustainable future. Air travel is more CO2-intensive than automobile transport (per-person-per-km). 

\item (Show slide of CO2 per mile traveled per person. Safety ratings. Numbers of “passengers.” Aggregate amounts of CO2 emitted.) Should we ban air travel? Why or why not? Are there ethical aspects to this question or only technical/practical? 

\item Whose opinion should we respect? (expert vs. novice)?

\item This policy question gets at several “axes” of decision-making.
\begin{itemize}
\item Economic gain vs. environmental harm
\item Individual harm vs. corporate good
\item Individual good vs. corporate harm
\item Benefits to current generation vs. future generations
\item Environmentally harmful but convenient (jet airliners) vs. Environmentally benign but inconvenient (blimps) (stewardship of time.)
\end{itemize}


\item Electric cars vs. FF cars. How equivalent do they have to be to … force adoption (ban FF)? What about as a historical artifact?

\item The proposed solution to the transportation CO2 emissions problem is to ban personal transport vehicles.

\item The proposed solution (banning personal transport vehicles) solves the environmental problem. But it may cause social and economic problems. What are they? 
\item How would each of the Christian traditions discussed in this chapter evaluate tradeoffs among environmental, social, and economic factors?
Suppose reducing the number of personal transport vehicles results in job loss for auto industry employees. How should Christians respond to unemployment that results from something “good” like saving the environment? Should there be a “social safety net?” How big should it be? Does your response come from your Christian worldview or your political orientation?  Does the answer depend on whether the job loss is the result of a technical innovation (“progress”) or a new regulation prohibiting something (e.g., DDT)?
\item The proposed solution implicitly accepts a conundrum: you can’t have personal transport vehicles and a flourishing environment. What assumptions are behind that conundrum? Can you think of alternative policies that break free from the conundrum and solve the dilemma? What are the social, economic, and environmental costs of those alternative policies?
\item The proposed solution (banning personal transport vehicles) will provide benefits for future generations (reduced CO2 concentrations) at the expense of convenience for the current generation. How do the Christian traditions described in this chapter understand intergenerational tradeoffs?
How do Christians decide whether to advocate for “ideal” solutions (banning all personal transport vehicles) or pragmatic alternatives? How should Christians decide? Give examples.
\item Any change, such as reducing carbon emissions by banning personal transport vehicles, results in consequences that have DIRECT dollar-measurable impacts, such as increased personal cost per mile traveled, INDIRECT but still dollar-measurable consequences, such as reduced CO2 output, and consequences that aren’t measurable in dollars, such as an aesthetic impact on the landscape. Economic cost-benefit analysis converts all social and environmental costs to dollars and compares total costs. How should Christians evaluate choices that have dollar costs and non-dollar costs? Similarly, Pareto optimality is used to conceptualize choices in a multiobjective environment. How do Christians think about tradeoffs under Pareto optimality? Should we be this mathematical in our decision making?
\item What about when cost and benefits fall to different (groups of) people? When many benefit at the expense of a few? When few benefit at the expense of the many? When benefits and costs accrue to different generations? Does the size of the benefit/cost matter? Does the number of people in the group matter? Are there guidelines on what’s a “big enough” difference to matter? Possible examples include eminent domain, corporate profit vs. air pollution, aquifer depletion, species extinction, and climate change. My living room window looks at a mine.
\item What’s the Christian solution to tragedy of the commons?
\item What are Christian perspectives on population growth and/or control? How are these different?
Christians should work to eliminate poverty. This is a teaching of both the Old and New Testaments. How do we respond to the observation that alleviating poverty increases economic consumption, which has negative environmental consequences?
\item I need to make a living versus the value of wilderness (“doing nothing”). Can I create another subsistence farm out of the Amazon rainforest?
\item How do the origins of the Christian justifications for private property inform the answers to these other questions?

\item Should we have all public transportation (mass transit) instead? (Maybe in a city center only.)
\end{enumerate}
%
Is anybody “doing theology” in this space?



\printbibliography


\end{document}