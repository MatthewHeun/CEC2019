\title{\textbf{A survey of Christian perspectives \\
		on sustainability and related topics}}
\author{
		% Authors must be italicized.
        \emph{Jeremy Van Antwerp} and \emph{Matthew Kuperus Heun}%
\footnote{
Engineering Department, 
Calvin College,
Grand Rapids, MI, 49546, USA}
}

\documentclass[12pt]{article}

% CEC papers should be set in Times New Roman.
% https://tex.stackexchange.com/questions/153168/how-to-set-document-font-to-times-new-roman-by-command
% suggests the following.
\usepackage{mathptmx}             % For Times New Roman font (or a close approximation thereof).
\usepackage[margin=1in]{geometry} % For 1-inch margins all around.
\usepackage[document]{ragged2e}   % For left justification.
\usepackage{parskip}              % For double-spacing between paragraphs.
\usepackage{nopageno}             % To eliminate page numbers.
% For MLA bibliography. See https://www.ctan.org/pkg/biblatex-mla?lang=en for details.
\usepackage[american]{babel} 
\usepackage{csquotes} 
\usepackage[style=mla-new]{biblatex}
\addbibresource{JVAMKH.bib}
\usepackage{titlesec}             % To change formats of section titles, etc.
\titleformat*{\section}{\normalsize}
\titleformat*{\subsection}{\normalsize}
\titleformat*{\subsubsection}{\normalsize}
\usepackage{abstract}             % Set characteristics of the abstract
\setlength{\absleftindent}{0in}   % Do not indent left side
\setlength{\absrightindent}{0in}  % Do not indent right side
\setlength{\parindent}{0in}       % Do not indent the 1st line of paragraphs.
\setlength{\parskip}{12pt}        % Instead, add space between paragraphs.
\renewcommand{\abstractnamefont}{\normalfont\normalsize} % Unbold and regular size
\renewcommand{\abstracttextfont}{\normalfont\normalsize} % Unbold and regular size
\date{}                           % To eliminate the date in the title


\begin{document}
	
\maketitle

\begin{abstract}
\noindent
%**** This is the abstract that I have on file. ---MKH ****
This paper presents a survey of Christian perspectives, from ancient to modern, on a
variety of sustainability-related topics such as stewardship of the natural world,
economic growth and technological change, energy, and human population. The emphasis is on
the development of thought and the lineage of ideas, with application to modern viewpoints
on current issues such as climate change, sustainable development, and human migration. We
use Willis Jenkins’ topology (Ecologies of Grace) to organize three modern strands
(ecojustice, stewardship, and “ecological spiritualities”) arising from three Christian
traditions (RomancCatholicism, reformed Christianity, and Eastern Orthodoxy,
respectively).

The paper will cover a broad range of Christian thought and teaching in a digestible and
coherent format. It will serve as a supplement to a future engineering textbook on
sustainability challenges. Textbook chapters will provide a platform of background
knowledge to facilitate one-hour in-class discussions of several sustainability topics or
challenges. The conference presentation will highlight one area of Christian thought
(stewardship) and focus on piloting classroom discussion questions related to the theology
of sustainability.

it works without breaking. Bogus line in abstract from MKH. This is another change on the same line.
\end{abstract}


%%%%%%%%%%%%%%%%%%%%%%%%%%%%%%%%%%%%%%%%%%%%%%%%%%%%%%%%%%%%%%
\section{Introduction}
\label{sec:introduction}
%%%%%%%%%%%%%%%%%%%%%%%%%%%%%%%%%%%%%%%%%%%%%%%%%%%%%%%%%%%%%%

Sustainability is a popular topic and is of increasing concern in the modern world. This
paper will begin to answer the questions ``how have Christians in the past thought about
sustainability topics?'' ```How do Christians today think about sustainability topics?''
and ``Where do these views come from?'' We will end with some questions for thought and
discussion.

Sustainability is often organized into three categories: environmental, economic, and
social. Environmental sustainability involves the idea of preserving the natural world.
Christians have historically understood this as stewardship of the natural world, going
back to the charge God gives humankind in Genesis 1:26. Economic sustainability refers to
the preservation and increase in value of human activities. In other words, not everything
we do can lose money. Social sustainability refers to relationships among humans -- that
there should be justice, peace, order, and flourishing in human society. For Christians,
the root of this concern is the command to ``love your neighbor as yourself.'' Overarching
these three areas are concerns for human shalom and wellbeing.

How Christians have understood these imperatives has varied considerably over millennia. 

%Paragraphs must be separated by a blank line.
%This paragraph has a citation~\autocite[42]{bogus}.


%1%%%%%%%%%%%%%%%%%%%%%%%%%%%%%%%%%%%%%%%%%%%%%%%%%%%%%%%%%%%%
\section{Environmental}
\label{sec:environmental}
%%%%%%%%%%%%%%%%%%%%%%%%%%%%%%%%%%%%%%%%%%%%%%%%%%%%%%%%%%%%%%

Natural world is inherently good.
Natural world should be conquered and ruled over (in part, because it is `red in tooth and claw.')
%Include table with links to denominational statements on climate change, whaling, conservation, pollution, wilderness?


%++++++++++++++++++++++++++++++
\subsection{Environmental history}
\label{sec:environmental_history}
%++++++++++++++++++++++++++++++

%2%%%%%%%%%%%%%%%%%%%%%%%%%%%%%%%%%%%%%%%%%%%%%%%%%%%%%%%%%%%%
\section{Economic}
\label{sec:economic}
%%%%%%%%%%%%%%%%%%%%%%%%%%%%%%%%%%%%%%%%%%%%%%%%%%%%%%%%%%%%%%

Wealth is evil and distracts us from God.
Wealth is a sign of God's blessing. 
Personal property. Western of ``Christian'' concepts.
The Enlightenment and the concept of personal freedom.
Development of corporations.
Theology of saving. Commerce, banking, financial trading.
Roman Catholic response(s) to communism?
Who owns land? Labor? Capital?

%++++++++++++++++++++++++++++++
\subsection{History of Christian economic thought}
\label{sec:economic_history}
%++++++++++++++++++++++++++++++



%3%%%%%%%%%%%%%%%%%%%%%%%%%%%%%%%%%%%%%%%%%%%%%%%%%%%%%%%%%%%%
\section{Social}
\label{sec:social}
%%%%%%%%%%%%%%%%%%%%%%%%%%%%%%%%%%%%%%%%%%%%%%%%%%%%%%%%%%%%%%

Suffering and justice. Care for the poor. 
Kings, democracy, dictators.
Medicine, water, health care. What are ``rights?''
What are the means of redress?
Just war? just use of ``power?''
Sphere sovereignty and the role of government? Church? family ... school...


%++++++++++++++++++++++++++++++
\subsection{History of Christian social thought}
\label{sec:social_history}
%++++++++++++++++++++++++++++++


\section{Other?}
\section{Conclusions?}



\printbibliography


\end{document}