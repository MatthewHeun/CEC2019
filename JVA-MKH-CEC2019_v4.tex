\title{\textbf{Christian perspectives on sustainability: \\
		The need for numerical answers to philosophical questions.}}
% Sustainability challenges and managing tradeoffs: engineers' need for numerical answers to philosophical questions 
\author{
		% Authors must be italicized.
        \emph{Jeremy Van Antwerp} and \emph{Matthew Kuperus Heun}%
\footnote{
Engineering Department, 
Calvin College,
Grand Rapids, MI, 49546, USA}
}

\documentclass[12pt]{article}

% CEC papers should be set in Times New Roman.
% https://tex.stackexchange.com/questions/153168/how-to-set-document-font-to-times-new-roman-by-command
% suggests the following.
\usepackage{mathptmx}             % For Times New Roman font (or a close approximation thereof).
\usepackage[margin=1in]{geometry} % For 1-inch margins all around.
\usepackage[document]{ragged2e}   % For left justification.
\usepackage{parskip}              % For double-spacing between paragraphs.
\usepackage{nopageno}             % To eliminate page numbers.
% For MLA bibliography. See https://www.ctan.org/pkg/biblatex-mla?lang=en for details.
\usepackage[american]{babel}
\usepackage{csquotes}
\usepackage[style=mla-new]{biblatex}
\addbibresource{JVAMKH.bib}

\usepackage{titlesec}             % To change formats of section titles, etc.
\titleformat*{\section}{\normalsize}
\titleformat*{\subsection}{\normalsize}
\titleformat*{\subsubsection}{\normalsize}
\titleformat*{\paragraph}{\itshape} % Set paragraph titles to italics
\usepackage{abstract}             % Set characteristics of the abstract
\setlength{\absleftindent}{0in}   % Do not indent left side
\setlength{\absrightindent}{0in}  % Do not indent right side
\usepackage{url}
\setlength{\parindent}{0in}       % Do not indent the 1st line of paragraphs.
\setlength{\parskip}{12pt}        % Instead, add space between paragraphs.
\renewcommand{\abstractnamefont}{\normalfont\normalsize} % Unbold and regular size
\renewcommand{\abstracttextfont}{\normalfont\normalsize} % Unbold and regular size
\date{}                           % To eliminate the date in the title

% To include graphics
\usepackage{graphicx}

% Commands for editing.

\usepackage{xcolor}            % For colored text
\usepackage[normalem]{ulem}    % For \sout command (strikethrough)

% From https://tex.stackexchange.com/questions/130623/crossing-out-using-different-colour,
% Change the \sout color to red
\newcommand{\redsout}{\bgroup\markoverwith{\textcolor{red}{\rule[0.5ex]{2pt}{0.4pt}}}\ULon}

% Use these versions to display changes.
\newcommand{\del}[1]{\textcolor{gray}{\redsout{#1}}}
\newcommand{\ins}[1]{\textcolor{red}{#1}}
\newcommand{\rev}[2]{\del{#1}\ins{#2}}

% Use these versions for a clean copy.
% \newcommand{\del}[1]{}
% \newcommand{\ins}[1]{#1}
% \newcommand{\rev}[2]{#2}



\begin{document}
	
\maketitle

\begin{abstract}
\noindent
\ins{rewrite abstract from scratch. Later.}

\end{abstract}


%%%%%%%%%%%%%%%%%%%%%%%%%%%%%%%%%%%%%%%%%%%%%%%%%%%%%%%%%%%%%%
\section{Sustainability and the disciplines}
\label{sec:sustainability_and_the_disciplines}
%%%%%%%%%%%%%%%%%%%%%%%%%%%%%%%%%%%%%%%%%%%%%%%%%%%%%%%%%%%%%%



%%%%%%%%%%%%%%%%%%%%%%%%%%%%%%%%%%%%%%%%%%%%%%%%%%%%%%%%%%%%%%
\section{Questions}
\label{sec:questions}
%%%%%%%%%%%%%%%%%%%%%%%%%%%%%%%%%%%%%%%%%%%%%%%%%%%%%%%%%%%%%%

\ins{flow from questions to gaps in how Christians think about sustainability.}
Each paragraph ends with a question (in italics) 
move the the ends of motivating question ==> We lack tools to give concrete answers to these questions.

\subsection{Social}
For any given level of technological development, there must exist some upper limit to the human population the Earth
can support. Calculating this limit is a technical question and can be technically addressed. The subsequent questions are nontechnical.
\begin{itemize}
\item ``What is the ideal total population?'' (which is less than or equal to maximum possible population) % informed by environmental (and economic?) tradeoffs.
\item ``How should we manage for and arrive at this ideal population?'' and % a means/ends question.
\item ``Where should people live? What is the geographical distribution of population?'' % rights, standard of living = justice.
\end{itemize}
Answers to these questions depend a great deal on \emph{values}. What are the relative preferences for justice, standard of living, 
environmental versus economic and social tradeoffs, and the relative importance of the means versus the end.


2. Regarding gloomy predictions about sustainability issues, such as climate change, global population,
or in terms of world energy supply, there are those who will say ``don't worry, it will all work out,'' while
others respond ``no it won't.'' Who should we give credence to? Obviously, those who are telling the truth. But that can
be hard to identify. (The Old Testament criteria for determining the legitimacy of a prophet comes to mind, but the
Israelites had a poor track record of following prophetic advice.) There are technical truths and moral truths. Should
the opinion of an expert researcher or technologist count for more than a blue-collar ``man on the street?'' How about a
government leader? Does it matter if the question under consideration is technical or nontechnical?

10. How do/should Christian assessments of tradeoffs change when cost and benefits fall to different (groups of) people?
When many benefit at the expense of a few? When few benefit at the expense of many? When benefits and costs accrue to
different generations? Does the size of the benefit/cost matter? Does the number of people in the group matter? 
Of course they do -- so what’s a ``big enough'' difference to matter?

12. What ``right'' do people have to food, water, air, medicine or health care, property, or a ``living'' wage? 
Where do these rights come from? How do legal rights shape moral rights and vice versa?

5. Questions about the legality of a ban on air travel would surely arise. Even for the benefit of the long-term survival
of life on planet Earth, would we make such a draconian decree? Is it ethical to do so?


\subsection{Social economics}

9. A policy-driven change, such as reducing carbon emissions by banning air travel, results in consequences that have
\emph{direct} dollar-measurable impacts (such as change in GDP), \emph{indirect} but still dollar-measurable
consequences (such as reduced CO$_2$ output), and consequences that \emph{aren't} measurable in dollars (such as an
aesthetic impact on the landscape). Economic cost-benefit analysis converts all social and environmental costs to
dollars, thereby providing a consistent numeraire and allowing direct comparisons among environmental, economic, and
social effects. How should Christians evaluate choices among policy solutions whose impacts are dollar-quantifiable and
those that are \emph{not} dollar-quantifiable? What is the value of ``wilderness?''



\subsection{Environmental social justice}
3. A Brazilian farmer argues that he/she needs to make a living. Can he/she create another subsistence farm in the Amazon rainforest?  
It could be argued that it is sinful \emph{not} to make use of a resource, such as coal or petroleum. What is the balance between
the needs to the present and the needs of the future?		
11. Both the Old and New Testaments teach that Christians should work to eliminate poverty. How do we respond to the 
observation that alleviating poverty increases economic consumption, which has negative environmental consequences?
How do we balance stewardship of the natural world with loving our neighbor?


4. In 2013, air travel was responsible for about 3\% of US greenhouse gas emissions. % citation needed. EPA data appears to be offline now.
% see https://www.c2es.org/content/reducing-carbon-dioxide-emissions-from-aircraft/
% which cites Inventory of U.S. Greenhouse Gas Emissions and Sinks: 1990–2015 (U.S. Environmental Protection Agency, 2017)
Air travel is a particularly \emph{carbon intensive} activity. Therefore, banning air travel would be a step to making the 
world a more sustainable place. Engineers are well equipped to analyze some of the effects of this proposal. For instance,
engineers could estimate the resulting \emph{increase} in emissions from other forms of transportation that would result from 
the ban on air travel. However, \emph{how should the social and economic consequences be weighed against the environmental 
benefits}? 


\subsection{Multifaceted sustainability}


6. Some consequences of a ban on air travel would include the destruction of the air travel industry and its ancillary
industries. How does the weighing of tradeoffs (e.g., economic versus environmental) differ if the loss of jobs is not
the result of a regulation, but instead comes from technological innovation or market forces? Should this make a
difference? Does it?

7. Banning air travel might be considered a ``clean'' or ``ideal'' policy option. Other policy options could be classified
as pragmatic, such as improving air travel energy efficiency. How should Christians navigate the space between ideal and
pragmatic policy proposals?

8. There are those who say that coal mining is good because the benefit of employment in the coal industry outweighs
any (potential) environmental harm caused from mining and burning coal. How large does the cost-benefit differential
have to be for this to be true? This is an entirely practical question that demands a concrete, numerical answer.


		  



As a society, we have \emph{de facto} arrived at answers to questions such as the ones above. However, 
there is increasing reason to believe that the answers we have are not the answers that will lead to a sustainable future.
Moreover, the hows and whys of the answers that we have need to be reexamined.
\ins{Need a concluding and/or summarizing statement. Something along the lines of ``thus, it's clear that we lack the theological
and philosophical framework that would allow us to to address sustainability issues.}

% Add: Utilitarian/Kantian ethics? Engineers like these because they allow us to calculate the greatest good for the greatest number of people.
% If an answer is mathematical, it must be true.


%%%%%%%%%%%%%%%%%%%%%%%%%%%%%%%%%%%%%%%%%%%%%%%%%%%%%%%%%%%%%%
\section{The need for a theology of sustainability}
\label{sec:need_for_theology_of_sustainability}
%%%%%%%%%%%%%%%%%%%%%%%%%%%%%%%%%%%%%%%%%%%%%%%%%%%%%%%%%%%%%%



%%%%%%%%%%%%%%%%%%%%%%%%%%%%%%%%%%%%%%%%%%%%%%%%%%%%%%%%%%%%%%
\section{Worldviews}
\label{sec:worldviews}
%%%%%%%%%%%%%%%%%%%%%%%%%%%%%%%%%%%%%%%%%%%%%%%%%%%%%%%%%%%%%%


%%%%%%%%%%%%%%%%%%%%%%%%%%%%%%%%%%%%%%%%%%%%%%%%%%%%%%%%%%%%%%
\section{Conclusions}
\label{sec:conclusions}
%%%%%%%%%%%%%%%%%%%%%%%%%%%%%%%%%%%%%%%%%%%%%%%%%%%%%%%%%%%%%%



\printbibliography
\end{document}