\title{\textbf{Christian perspectives on sustainability: \\
		The need for numerical answers to philosophical questions.}}
% Sustainability challenges and managing tradeoffs: engineers' need for numerical answers to philosophical questions 
\author{
		% Authors must be italicized.
        \emph{Jeremy Van Antwerp} and \emph{Matthew Kuperus Heun}%
\footnote{
Engineering Department, 
Calvin College,
Grand Rapids, MI, 49546, USA}
}

\documentclass[12pt]{article}

% CEC papers should be set in Times New Roman.
% https://tex.stackexchange.com/questions/153168/how-to-set-document-font-to-times-new-roman-by-command
% suggests the following.
\usepackage{mathptmx}             % For Times New Roman font (or a close approximation thereof).
\usepackage[margin=1in]{geometry} % For 1-inch margins all around.
\usepackage[document]{ragged2e}   % For left justification.
\usepackage{parskip}              % For double-spacing between paragraphs.
\usepackage{nopageno}             % To eliminate page numbers.
% For MLA bibliography. See https://www.ctan.org/pkg/biblatex-mla?lang=en for details.
\usepackage[american]{babel}
\usepackage{csquotes}
\usepackage[style=mla-new]{biblatex}
\addbibresource{JVAMKH.bib}

\usepackage{titlesec}             % To change formats of section titles, etc.
\titleformat*{\section}{\normalsize}
\titleformat*{\subsection}{\normalsize}
\titleformat*{\subsubsection}{\normalsize}
\titleformat*{\paragraph}{\itshape} % Set paragraph titles to italics
\usepackage{abstract}             % Set characteristics of the abstract
\setlength{\absleftindent}{0in}   % Do not indent left side
\setlength{\absrightindent}{0in}  % Do not indent right side
\usepackage{url}
\setlength{\parindent}{0in}       % Do not indent the 1st line of paragraphs.
\setlength{\parskip}{12pt}        % Instead, add space between paragraphs.
\renewcommand{\abstractnamefont}{\normalfont\normalsize} % Unbold and regular size
\renewcommand{\abstracttextfont}{\normalfont\normalsize} % Unbold and regular size
\date{}                           % To eliminate the date in the title

% To include graphics
\usepackage{graphicx}

% Commands for editing.

\usepackage{xcolor}            % For colored text
\usepackage[normalem]{ulem}    % For \sout command (strikethrough)

% From https://tex.stackexchange.com/questions/130623/crossing-out-using-different-colour,
% Change the \sout color to red
\newcommand{\redsout}{\bgroup\markoverwith{\textcolor{red}{\rule[0.5ex]{2pt}{0.4pt}}}\ULon}

% Use these versions to display changes.
\newcommand{\del}[1]{\textcolor{gray}{\redsout{#1}}}
\newcommand{\ins}[1]{\textcolor{red}{#1}}
\newcommand{\rev}[2]{\del{#1}\ins{#2}}

% Use these versions for a clean copy.
% \newcommand{\del}[1]{}
% \newcommand{\ins}[1]{#1}
% \newcommand{\rev}[2]{#2}



\begin{document}
	
\maketitle

\begin{abstract}
\noindent
\ins{rewrite abstract from scratch. Later.}

\end{abstract}


%%%%%%%%%%%%%%%%%%%%%%%%%%%%%%%%%%%%%%%%%%%%%%%%%%%%%%%%%%%%%%
\section{Sustainability and the disciplines}
\label{sec:sustainability_and_the_disciplines}
%%%%%%%%%%%%%%%%%%%%%%%%%%%%%%%%%%%%%%%%%%%%%%%%%%%%%%%%%%%%%%

Because of the circumstance of our world, 
sustainability is a significant concern. 
Answers to sustainability questions are corollarycritically needed.
However, sustainability challenges are ``wicked problems,'' %citation needed.
in part because they are complex, multidisciplinary, and multifaceted. 
Sustainability has many aspects that are concrete, technical challenges,
and so engineers have an important role to play in the transition to a sustainable future.
At the same time, there are many aspects of sustainability challenges that are 
nontechnical, even philosophical, yet still require concrete quantitative answers.
As we will show, humans largely lack a framework for answering these questions. 
It is the aim of this paper to motivate Christian engineers to begin to provide numerical answers
to philosophical questions.



\begin{figure}
\centering
\includegraphics[width=1\linewidth]{figure_other/TriangleDiagram.pdf}
\caption{Three aspects of sustainability.}
\label{fig:3_sustain}
\end{figure}

%%%%%%%%%%%%%%%%%%%%%%%%%%%%%%%%%%%%%%%%%%%%%%%%%%%%%%%%%%%%%%
\section{Questions}
\label{sec:questions}
%%%%%%%%%%%%%%%%%%%%%%%%%%%%%%%%%%%%%%%%%%%%%%%%%%%%%%%%%%%%%%



%%%%%%%%%%%%%%%%%%%%%%%%%%%%%%%%%%%%%%%%%%%%%%%%%%%%%%%%%%%%%%
\section{The need for a theology of sustainability}
\label{sec:need_for_theology_of_sustainability}
%%%%%%%%%%%%%%%%%%%%%%%%%%%%%%%%%%%%%%%%%%%%%%%%%%%%%%%%%%%%%%



%%%%%%%%%%%%%%%%%%%%%%%%%%%%%%%%%%%%%%%%%%%%%%%%%%%%%%%%%%%%%%
\section{Worldviews}
\label{sec:worldviews}
%%%%%%%%%%%%%%%%%%%%%%%%%%%%%%%%%%%%%%%%%%%%%%%%%%%%%%%%%%%%%%


%%%%%%%%%%%%%%%%%%%%%%%%%%%%%%%%%%%%%%%%%%%%%%%%%%%%%%%%%%%%%%
\section{Conclusions}
\label{sec:conclusions}
%%%%%%%%%%%%%%%%%%%%%%%%%%%%%%%%%%%%%%%%%%%%%%%%%%%%%%%%%%%%%%



\printbibliography
\end{document}